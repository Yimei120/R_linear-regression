% Options for packages loaded elsewhere
\PassOptionsToPackage{unicode}{hyperref}
\PassOptionsToPackage{hyphens}{url}
%
\documentclass[
]{article}
\usepackage{lmodern}
\usepackage{amssymb,amsmath}
\usepackage{ifxetex,ifluatex}
\ifnum 0\ifxetex 1\fi\ifluatex 1\fi=0 % if pdftex
  \usepackage[T1]{fontenc}
  \usepackage[utf8]{inputenc}
  \usepackage{textcomp} % provide euro and other symbols
\else % if luatex or xetex
  \usepackage{unicode-math}
  \defaultfontfeatures{Scale=MatchLowercase}
  \defaultfontfeatures[\rmfamily]{Ligatures=TeX,Scale=1}
\fi
% Use upquote if available, for straight quotes in verbatim environments
\IfFileExists{upquote.sty}{\usepackage{upquote}}{}
\IfFileExists{microtype.sty}{% use microtype if available
  \usepackage[]{microtype}
  \UseMicrotypeSet[protrusion]{basicmath} % disable protrusion for tt fonts
}{}
\makeatletter
\@ifundefined{KOMAClassName}{% if non-KOMA class
  \IfFileExists{parskip.sty}{%
    \usepackage{parskip}
  }{% else
    \setlength{\parindent}{0pt}
    \setlength{\parskip}{6pt plus 2pt minus 1pt}}
}{% if KOMA class
  \KOMAoptions{parskip=half}}
\makeatother
\usepackage{xcolor}
\IfFileExists{xurl.sty}{\usepackage{xurl}}{} % add URL line breaks if available
\IfFileExists{bookmark.sty}{\usepackage{bookmark}}{\usepackage{hyperref}}
\hypersetup{
  pdftitle={STOR 565 Spring 2020 Homework 2},
  pdfauthor={Yimei Fan},
  hidelinks,
  pdfcreator={LaTeX via pandoc}}
\urlstyle{same} % disable monospaced font for URLs
\usepackage[margin=1in]{geometry}
\usepackage{color}
\usepackage{fancyvrb}
\newcommand{\VerbBar}{|}
\newcommand{\VERB}{\Verb[commandchars=\\\{\}]}
\DefineVerbatimEnvironment{Highlighting}{Verbatim}{commandchars=\\\{\}}
% Add ',fontsize=\small' for more characters per line
\usepackage{framed}
\definecolor{shadecolor}{RGB}{248,248,248}
\newenvironment{Shaded}{\begin{snugshade}}{\end{snugshade}}
\newcommand{\AlertTok}[1]{\textcolor[rgb]{0.94,0.16,0.16}{#1}}
\newcommand{\AnnotationTok}[1]{\textcolor[rgb]{0.56,0.35,0.01}{\textbf{\textit{#1}}}}
\newcommand{\AttributeTok}[1]{\textcolor[rgb]{0.77,0.63,0.00}{#1}}
\newcommand{\BaseNTok}[1]{\textcolor[rgb]{0.00,0.00,0.81}{#1}}
\newcommand{\BuiltInTok}[1]{#1}
\newcommand{\CharTok}[1]{\textcolor[rgb]{0.31,0.60,0.02}{#1}}
\newcommand{\CommentTok}[1]{\textcolor[rgb]{0.56,0.35,0.01}{\textit{#1}}}
\newcommand{\CommentVarTok}[1]{\textcolor[rgb]{0.56,0.35,0.01}{\textbf{\textit{#1}}}}
\newcommand{\ConstantTok}[1]{\textcolor[rgb]{0.00,0.00,0.00}{#1}}
\newcommand{\ControlFlowTok}[1]{\textcolor[rgb]{0.13,0.29,0.53}{\textbf{#1}}}
\newcommand{\DataTypeTok}[1]{\textcolor[rgb]{0.13,0.29,0.53}{#1}}
\newcommand{\DecValTok}[1]{\textcolor[rgb]{0.00,0.00,0.81}{#1}}
\newcommand{\DocumentationTok}[1]{\textcolor[rgb]{0.56,0.35,0.01}{\textbf{\textit{#1}}}}
\newcommand{\ErrorTok}[1]{\textcolor[rgb]{0.64,0.00,0.00}{\textbf{#1}}}
\newcommand{\ExtensionTok}[1]{#1}
\newcommand{\FloatTok}[1]{\textcolor[rgb]{0.00,0.00,0.81}{#1}}
\newcommand{\FunctionTok}[1]{\textcolor[rgb]{0.00,0.00,0.00}{#1}}
\newcommand{\ImportTok}[1]{#1}
\newcommand{\InformationTok}[1]{\textcolor[rgb]{0.56,0.35,0.01}{\textbf{\textit{#1}}}}
\newcommand{\KeywordTok}[1]{\textcolor[rgb]{0.13,0.29,0.53}{\textbf{#1}}}
\newcommand{\NormalTok}[1]{#1}
\newcommand{\OperatorTok}[1]{\textcolor[rgb]{0.81,0.36,0.00}{\textbf{#1}}}
\newcommand{\OtherTok}[1]{\textcolor[rgb]{0.56,0.35,0.01}{#1}}
\newcommand{\PreprocessorTok}[1]{\textcolor[rgb]{0.56,0.35,0.01}{\textit{#1}}}
\newcommand{\RegionMarkerTok}[1]{#1}
\newcommand{\SpecialCharTok}[1]{\textcolor[rgb]{0.00,0.00,0.00}{#1}}
\newcommand{\SpecialStringTok}[1]{\textcolor[rgb]{0.31,0.60,0.02}{#1}}
\newcommand{\StringTok}[1]{\textcolor[rgb]{0.31,0.60,0.02}{#1}}
\newcommand{\VariableTok}[1]{\textcolor[rgb]{0.00,0.00,0.00}{#1}}
\newcommand{\VerbatimStringTok}[1]{\textcolor[rgb]{0.31,0.60,0.02}{#1}}
\newcommand{\WarningTok}[1]{\textcolor[rgb]{0.56,0.35,0.01}{\textbf{\textit{#1}}}}
\usepackage{graphicx,grffile}
\makeatletter
\def\maxwidth{\ifdim\Gin@nat@width>\linewidth\linewidth\else\Gin@nat@width\fi}
\def\maxheight{\ifdim\Gin@nat@height>\textheight\textheight\else\Gin@nat@height\fi}
\makeatother
% Scale images if necessary, so that they will not overflow the page
% margins by default, and it is still possible to overwrite the defaults
% using explicit options in \includegraphics[width, height, ...]{}
\setkeys{Gin}{width=\maxwidth,height=\maxheight,keepaspectratio}
% Set default figure placement to htbp
\makeatletter
\def\fps@figure{htbp}
\makeatother
\setlength{\emergencystretch}{3em} % prevent overfull lines
\providecommand{\tightlist}{%
  \setlength{\itemsep}{0pt}\setlength{\parskip}{0pt}}
\setcounter{secnumdepth}{-\maxdimen} % remove section numbering
\usepackage{amsgen,amsmath,amstext,amsbsy,amsopn,amssymb,mathabx,amsthm,bm,bbm}

\title{STOR 565 Spring 2020 Homework 2}
\usepackage{etoolbox}
\makeatletter
\providecommand{\subtitle}[1]{% add subtitle to \maketitle
  \apptocmd{\@title}{\par {\large #1 \par}}{}{}
}
\makeatother
\subtitle{\textbf{Due on 01/27/2020 in Class}}
\author{Yimei Fan}
\date{}

\begin{document}
\maketitle

\theoremstyle{definition}
\newtheorem*{hint}{Hint}

\theoremstyle{remark}
\newtheorem*{rmk}{Remark}

\emph{Remark.} This homework aims to help you go through the necessary
preliminary material from linear regression. Credits for
\textbf{Theoretical Part} and \textbf{Computational Part} are in total
110 pt.~At most 100 pt can be earned for this homework. For
\textbf{Computational Part}, please complete your answer in the
\textbf{RMarkdown} file and submit your printed PDF homework created
using \textbf{RMarkdown}.

\hypertarget{computational-part}{%
\subsection{Computational Part}\label{computational-part}}

\begin{enumerate}
\def\labelenumi{\arabic{enumi}.}
\tightlist
\item
  (\emph{25 pt}) Consider the dataset ``Boston'' in predicting the crime
  rate at Boston with associated covariates.
\end{enumerate}

\begin{Shaded}
\begin{Highlighting}[]
\KeywordTok{head}\NormalTok{(Boston)}
\end{Highlighting}
\end{Shaded}

\begin{verbatim}
##      crim zn indus chas   nox    rm  age    dis rad tax ptratio  black lstat
## 1 0.00632 18  2.31    0 0.538 6.575 65.2 4.0900   1 296    15.3 396.90  4.98
## 2 0.02731  0  7.07    0 0.469 6.421 78.9 4.9671   2 242    17.8 396.90  9.14
## 3 0.02729  0  7.07    0 0.469 7.185 61.1 4.9671   2 242    17.8 392.83  4.03
## 4 0.03237  0  2.18    0 0.458 6.998 45.8 6.0622   3 222    18.7 394.63  2.94
## 5 0.06905  0  2.18    0 0.458 7.147 54.2 6.0622   3 222    18.7 396.90  5.33
## 6 0.02985  0  2.18    0 0.458 6.430 58.7 6.0622   3 222    18.7 394.12  5.21
##   medv
## 1 24.0
## 2 21.6
## 3 34.7
## 4 33.4
## 5 36.2
## 6 28.7
\end{verbatim}

Suppose you would like to predict the crime rate with explantory
variables

\begin{itemize}
\tightlist
\item
  \texttt{age} - Proportion of owner-occupied units built prior to 1940
\item
  \texttt{dis} - Weighted mean of distances to employement centers
\item
  \texttt{rad} - Index of accessibility to radial highways
\end{itemize}

Run with the linear model

\begin{Shaded}
\begin{Highlighting}[]
\NormalTok{mod1 <-}\StringTok{ }\KeywordTok{lm}\NormalTok{(crim }\OperatorTok{~}\StringTok{ }\NormalTok{age }\OperatorTok{+}\StringTok{ }\NormalTok{dis }\OperatorTok{+}\StringTok{ }\NormalTok{rad, }\DataTypeTok{data =}\NormalTok{ Boston)}
\KeywordTok{summary}\NormalTok{(mod1)}
\end{Highlighting}
\end{Shaded}

\begin{verbatim}
## 
## Call:
## lm(formula = crim ~ age + dis + rad, data = Boston)
## 
## Residuals:
##    Min     1Q Median     3Q    Max 
## -9.097 -1.572 -0.265  0.699 76.240 
## 
## Coefficients:
##             Estimate Std. Error t value Pr(>|t|)    
## (Intercept) -1.75072    1.85563  -0.943    0.346    
## age          0.01339    0.01611   0.831    0.406    
## dis         -0.25642    0.22056  -1.163    0.246    
## rad          0.56750    0.03979  14.264   <2e-16 ***
## ---
## Signif. codes:  0 '***' 0.001 '**' 0.01 '*' 0.05 '.' 0.1 ' ' 1
## 
## Residual standard error: 6.69 on 502 degrees of freedom
## Multiple R-squared:  0.3986, Adjusted R-squared:  0.395 
## F-statistic: 110.9 on 3 and 502 DF,  p-value: < 2.2e-16
\end{verbatim}

Answer the following questions.

\begin{enumerate}
\def\labelenumi{(\roman{enumi})}
\item
  What do the following quantities that appear in the above output mean
  in the linear model? Provide a brief description.

  \begin{itemize}
  \tightlist
  \item
    \texttt{t\ value} and
    \texttt{Pr(\textgreater{}\textbar{}t\textbar{})} of \texttt{rad}
  \end{itemize}

  \textbf{Answer:} YOUR ANSWER. t-value: it is calculated by
  estimate/Std.Error. It is used to calculated P-value, which can be
  used to do hypothesis testing for each variables.
  \texttt{Pr(\textgreater{}\textbar{}t\textbar{})} of \texttt{rad}: H0:
  the coefficient of `rad' equals to zero. Ha: the coefficient of `rad'
  does not equal to zero. Since the p-value for rad is \textless2e-16,
  and it is lower than 0.05. We can reject H0, which means the
  coefficient of rad is not zero and it is significant.

  \begin{center}\rule{0.5\linewidth}{\linethickness}\end{center}

  \begin{itemize}
  \tightlist
  \item
    \texttt{Adjusted\ R-squared}
  \end{itemize}

  \textbf{Answer:} YOUR ANSWER. Adjusted R-squared: It measures how well
  the model can fit the data. It will not be influenced by the number of
  variables. The formula is 1-MSE/MST.

  \begin{center}\rule{0.5\linewidth}{\linethickness}\end{center}

  \begin{itemize}
  \tightlist
  \item
    \texttt{F-statistic}, \texttt{DF} and corresponding \texttt{p-value}
  \end{itemize}

  \textbf{Answer:} YOUR ANSWER. F-statistic: It is similar to
  t-statistic but F-statistic is for the whole model. If there is only
  one variable, then F-statistic should be the same with t-statistic. It
  compares the fit of the whole model and null model(Y=1) and the larger
  F-statistic is, the more significant the whole model will be. DF:
  There are two DF in the table. The first one is the number of
  variables(p), and the second one is for the degree of fredom for whole
  model: n-p-1. \texttt{p-value}: It is calculated from F-statistics,
  and it is used to conduct hypothesis test for H0: all coefficients are
  zero. Ha: At lease one coefficient is not zero.2.2e-16 is lower than
  0.05, therefore, we should reject H0, that means at least one
  coefficient is not zero.

  \begin{center}\rule{0.5\linewidth}{\linethickness}\end{center}
\item
  Are the following sentences True of False? Briefly justify your
  answer. + \texttt{age} is not a significant predictor of crim, and we
  can drop it from the model.
\end{enumerate}

\begin{verbatim}
**Answer:** YOUR ANSWER.
True. P-value is 0.406, and it is higher than 0.05.

***
+ Both `Multiple R-squared` and `Adjusted R-squared` increase with number of variables since they take into account all the variables and more variance is explained as the model becomes more complicated when the number of variables increases. 

**Answer:** YOUR ANSWER.
False. Adjusted R-squared will not increase with the number of variables, but it will increase if the model fits data better.

***    
+ `rad` has a positive effect on the response.

**Answer:** YOUR ANSWER.
True. The coefficient of 'rad' is 0.56750, and it is positive. Moreover, 'rad' is significant in this model, which means it does have effect on the response.


***
+ Our model residuals appear to be normally distributed.

\begin{hint}
  You need to access to the model residuals in justifying the last sentence. The following commands might help.
\end{hint}

```r
# Obtain the residuals
res1 <- residuals(mod1)

# Normal QQ-plot of residuals
plot(mod1, 2)

# Conduct a Normality test via Shapiro-Wilk and Kolmogorov-Smirnov test
shapiro.test(res1)
ks.test(res1, "pnorm")
```

**Answer:** YOUR ANSWER.

True. According to the qq-plot, the residuals roughly form a line, which means they are normally distributed.
According to Normality tests, W and D are larger than 0.05. Therefore, the residuals are normally distributed.

***
\end{verbatim}

\begin{enumerate}
\def\labelenumi{\arabic{enumi}.}
\setcounter{enumi}{1}
\tightlist
\item
  (\emph{30 pt}, Similar to Textbook Exercises 3.10) This question
  should be answered using the \texttt{Carseats} data set.
\end{enumerate}

\begin{Shaded}
\begin{Highlighting}[]
\KeywordTok{head}\NormalTok{(Carseats)}
\end{Highlighting}
\end{Shaded}

\begin{verbatim}
##   Sales CompPrice Income Advertising Population Price ShelveLoc Age Education
## 1  9.50       138     73          11        276   120       Bad  42        17
## 2 11.22       111     48          16        260    83      Good  65        10
## 3 10.06       113     35          10        269    80    Medium  59        12
## 4  7.40       117    100           4        466    97    Medium  55        14
## 5  4.15       141     64           3        340   128       Bad  38        13
## 6 10.81       124    113          13        501    72       Bad  78        16
##   Urban  US
## 1   Yes Yes
## 2   Yes Yes
## 3   Yes Yes
## 4   Yes Yes
## 5   Yes  No
## 6    No Yes
\end{verbatim}

\begin{enumerate}
\def\labelenumi{(\alph{enumi})}
\tightlist
\item
  Fit a multiple regression model to predict \texttt{Sales} using
  \texttt{US}, \texttt{Urban}, \texttt{Advertising} and \texttt{Price}.
\end{enumerate}

\textbf{Answer:} YOUR ANSWER. model =
lm(Sales\textasciitilde US+Urban+Advertising+Price, data = Carseats)
print(summary(model))

\begin{center}\rule{0.5\linewidth}{\linethickness}\end{center}

\begin{enumerate}
\def\labelenumi{(\alph{enumi})}
\setcounter{enumi}{1}
\tightlist
\item
  Provide an interpretation of each coefficient in the model. Be careful
  that some of the variables in the model are qualitative!
\end{enumerate}

\textbf{Answer:} YOUR ANSWER. The coeffieicents of `US' and `Urban' are
not significant, and we should not include them in the model. The
coefficient of `Advertising' is 0.120333, and it means that with more
advertising, the sales will increase. The coefficient of `Price' is
-0.054612, and it means if the price of car seats increase, the sales
will decrease accordingly.

\begin{center}\rule{0.5\linewidth}{\linethickness}\end{center}

\begin{enumerate}
\def\labelenumi{(\alph{enumi})}
\setcounter{enumi}{2}
\tightlist
\item
  Write out the model in equation form, being careful to handle the
  qualitative variables properly.
\end{enumerate}

\textbf{Answer:} YOUR ANSWER. If `Yes' in `US' and `Yes' in `Urban'
equals 1: sales = 13.03 + 0.12Advertising -0.05Price

If `Yes' in `US' and `Yes' in `Urban' equals 0: sales = 13.01 +
0.12Advertising -0.05Price

\begin{center}\rule{0.5\linewidth}{\linethickness}\end{center}

\begin{enumerate}
\def\labelenumi{(\alph{enumi})}
\setcounter{enumi}{3}
\tightlist
\item
  For which of the predictors can you reject the null hypothesis
  \(H_0 : \beta_j = 0\)?
\end{enumerate}

\textbf{Answer:} YOUR ANSWER. In order to reject the null hypothesis,
the P-value need to be smaller than 0.05. Therefore, for `Advertising'
and `Price', we can reject the null hypothesis.

\begin{center}\rule{0.5\linewidth}{\linethickness}\end{center}

\begin{enumerate}
\def\labelenumi{(\alph{enumi})}
\setcounter{enumi}{4}
\tightlist
\item
  On the basis of your response to the previous question, fit a smaller
  model that only uses the predictors for which there is evidence of
  association with the outcome.
\end{enumerate}

\textbf{Answer:} YOUR ANSWER.
summary(lm(Sales\textasciitilde Advertising+Price, data = Carseats))

\begin{center}\rule{0.5\linewidth}{\linethickness}\end{center}

\begin{enumerate}
\def\labelenumi{(\alph{enumi})}
\setcounter{enumi}{5}
\tightlist
\item
  How well do the models in (a) and (e) fit the data?
\end{enumerate}

\textbf{Answer:} YOUR ANSWER. The model in (a): Adjusted R-squared is
0.2747, P-value \textless{} 2.2e-16 The model in (e): Adjusted R-squared
is 0.2782, P-value \textless{} 2.2e-16

Compared to model(a), model(e) has slightly higher adjusted R-squared,
and both models are significant. Therefore, both model fit the data well
and model(e) is slightly better.

\begin{center}\rule{0.5\linewidth}{\linethickness}\end{center}

\begin{enumerate}
\def\labelenumi{(\alph{enumi})}
\setcounter{enumi}{6}
\tightlist
\item
  Using the model from (e), obtain 90\% confidence intervals for the
  coefficient(s).
\end{enumerate}

\textbf{Answer:} YOUR ANSWER. \#Advertising: tstar = qt(.9+0.05,397)

0.123107+ tstar\emph{0.018079}c(-1,1)

\#Price: tstarr = qt(.9+0.05,397)

-0.054613 + tstarr\emph{0.005078}c(-1,1)

\begin{center}\rule{0.5\linewidth}{\linethickness}\end{center}

\begin{enumerate}
\def\labelenumi{(\alph{enumi})}
\setcounter{enumi}{7}
\tightlist
\item
  Using the leave-one-out cross-validation and 5-fold cross-validation
  techniques to compare the performance of models in (a) and (e). What
  can you tell from (f) and (h)?
\end{enumerate}

\textbf{Hint.} Functions \texttt{update} (with option \texttt{subset})
and \texttt{predict}.

\textbf{Answer:} YOUR ANSWER.

library(caret)

\#leave-one-out cross-validation \#(a) train.control1 \textless-
trainControl(method = ``LOOCV'') model1 \textless-
train(Sales\textasciitilde US+Urban+Advertising+Price, data = Carseats,
method = ``lm'', trControl = train.control1) print(model1) \#(e)
train.control2 \textless- trainControl(method = ``LOOCV'') model2
\textless- train(Sales \textasciitilde{} Advertising + Price, data =
Carseats, method = ``lm'', trControl = train.control2) print(model2)

\#RMSE for (a) is 2.420802, and Rsquared is 0.263687; RMSE for (e) is
2.408753 and Rsquared is 0.2708067. (e) performs better.

\#5-fold cross-validation \#(a) set.seed(123) train.control3 \textless-
trainControl(method = ``cv'', number = 5) model3 \textless-
train(Sales\textasciitilde US+Urban+Advertising+Price, data = Carseats,
method = ``lm'', trControl = train.control3) print(model3)

\#(e) set.seed(123) train.control4 \textless- trainControl(method =
``cv'', number = 5) model4 \textless- train(Sales \textasciitilde{}
Advertising + Price, data = Carseats, method = ``lm'', trControl =
train.control4) print(model4)

\#RMSE for (a) is 2.436002, and Rsquared is 0.2746255; RMSE for (e) is
2.42669 and Rsquared is 0.2803708. (e) performs better.

\#(f) and (h) gives us similar result regarding to model performance and
they provide similar R-squared value.

\begin{center}\rule{0.5\linewidth}{\linethickness}\end{center}

\end{document}
